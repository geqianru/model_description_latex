\documentclass[preprint,12pt]{elsarticle}
\usepackage{amsthm}
\usepackage[english]{babel}
\usepackage{amssymb}
\usepackage{amsmath}
\usepackage{hyperref}
\usepackage{latexsym}
\usepackage{morefloats}
\usepackage{graphicx}
\usepackage{caption}
\usepackage{subcaption}
\usepackage{rotating}
\usepackage[linesnumbered,ruled,vlined]{algorithm2e}
\usepackage[nodots]{numcompress}
\DeclareMathOperator*{\argmin}{\arg\!\min}

\begin{document}
\nocite{*}

\begin{frontmatter}
\author{Qianru Ge}
\author{Hao Peng}
\author{Geert-Jan van Houtum}
\end{frontmatter}

\section{Introduction}

Capital goods are machines or products that are used by manufacturers to produce their end-products or
that are used by service organizations to deliver their services (van Houtum 2010?). Advanced technical systems such as medical systems, manufacturing systems, defense systems are capital goods which are critical for the operational processes of their customers. System downtime of these capital goods can cause serious consequences (e.g. millions of euros of reduced production output, extra waiting time of passengers, failure of military missions). Therefore, customers of these complex systems like hospitals, militaries and factories require a high system availability of these systems. Because of the complexity of the technology involved in capital goods, after-sale services such as maintenances and repairs is asked by the customers from the original equipment manufacturers (OEMs).

As a result, integrating services as a major sustainable source of the profit of OEMs has been widely recommended by a large amount of papers in recent decades. With after-sale services market has been growing to constitute a significant part of OEMs revenue, for example, a study conducted by Accenture (Dennis and Kambil 2003) shows that after-sale services contribute only 25\% of revenues across all manufacturing companies, but are responsible for $40\%-50\%$ of profits, many OEMs has been transforming their business strategy from product-oriented to service-oriented. After selling a system, under the traditional material-based contract, the OEM is responsible for the repair of the system only within the warranty period, which is often short compared to the lifetime of the system. After the warranty period, the OEM will charge the customer for providing spare parts, maintenance, and other services to keep the availability of the systems above certain levels. This may lead to higher spare parts costs, repair costs, and labor costs, and to lower system availabilities than a customer anticipated when buying the system. This undesired situation for the customer can be avoided by making better agreements with the OEM when a new system is being bought. A type of contract that may be attractive for both customers and OEMs is a Performance-Based Logistics (PBL) contract. Under a PBL contract, the OEM is responsible to meet a predetermined system availability target during a period of 5-10 years, say, and a penalty cost has to be paid to the customer when one fails to meet the system availability target. When designing the system, such system availability targets have to be taken into account. To increase system availability, optimal reliability design, which aims to get an optimal system-level configuration while considering the tradeoff between system performance and costs, via enhancing component reliability and building in redundancy (Kuo and Wan 2007), has received many attentions since the 1960s.

On the other hand, during the design phase of a system, engineers have to select a certain design among all the possible alternatives for each critical component in the system. In real life, the outcome of any development process for a certain design is uncertain with respect to the reliability requirement (Hussain and Murthy 2003). For example, since the failure mechanisms of some emerging technologies (e.g., Micro-Electro-Mechanical Systems) are complex, it is often difficult to predict the actual reliability behaviors of the critical components. Therefore newly-designed devices have been found to have uncertain component reliabilities, e.g., Sandia’s micro engine made by SUMMiTTM process (Figure 1). The uncertainty in component reliabilities can lead to large deviations of the realized system reliability from the expected system reliability (a point estimate for the system availability). In this case, if the optimization models for system reliability/redundancy design do not consider the uncertainty in component reliabilities, the obtained solutions may be poor solutions.

%According to Oliva and Kallenberg (2003), providing after-sale services not only becomes a competitive advantage of OEMs under global competitions, also satisfies customers increasing demands for more services. Since in complex technical systems such as medical systems, material handling systems, defense systems and manufacturing systems, system failures lead to millions of euros of lost production for the customer, it is important that the availability of the systems is high, and that down-situations are recovered quickly (Kranenburg and van Houtum, 2009).

In this paper, we developed a model to minimize the Life Cycle Cost (LCC) of a system, which is affected by the system availability, in terms of choosing different designs with uncertain parameters for each critical component. We investigate a situation in which an OEM will sell a complex series system with $m$ critical components to a customer together with an PBL contract which will cover the life time of the system. In the design phase of the system, the OEM should chose the optimal design for each critical component from a set of designs with uncertain reliability distribution parameters. We provide a method to compare different component designs at the single component level, providing the conditions under which one design dominates the orders. In addition, we constitute an efficient frontier that reflects the trade-off between the total costs and system availability.



The remainder of the paper is organized as follows:





\section{Literature}

\section{Model}
An OEM is designing a series system comprising of $m$ critical components for a customer. The system and its components has only two states: failed or operational. The life times of the components are independent exponentially distributed. The OEM cooperates with some secondary suppliers for the design of the components. Each secondary supplier has a rough design for certain component. Components from different rough designs have different failure rates and different prices. The lifetime of the system is estimated to be $T$. And the system will be sold with a performance-based contract for the estimated lifetime of the systems $T$. Within the performance-based contract period, the system availability should be above a predetermined level. A penalty cost will be paid to compensate the customer if the downtime exceeds the predetermined level. The OEM is responsible for all the failures. The LCC of the systems consists of three categories: (a) acquisition cost, (b) repair cost, (c) penalty cost.

%\includegraphics[width=60mm]{1.jpg}
   \subsection{Notations}
	 %\begin{center}
   \begin{tabular}{l l}
   %\hline
$i$ & index for critical components in the system\\
$j$ & index for rough designs of each component\\
$r$ & repair time\\
$T$ & the time length of the service contract period\\
$D_0$ & predetermined total downtime of a service contract\\
$c_r^{ij}$ & repair cost per time\\
$c_p$ & penalty cost per time unit after total downtime exceeding $D_0$ \\
$x_{ij}$ & decision variable\\
$\Lambda$    & system failure rate\\
$\Lambda_{i}$  & failure rate of the $i^{th}$ component in the system \\
$\Lambda_{ij}$ & failure rate of the $i^{th}$ component brought from supplier $j$ \\
$A(x_{ij})$  & total acquisition cost of the system\\
$c^{ij}_{a}$ & acquisition cost of the $i^{th}$ component brought from supplier $j$\\
$N(x_{ij})$ & expected number of repairs occur in period $[0, T]$\\
$\mu_{i}$ & mean of the failure rate in component $i$ \\
$\mu_{ij}$ & mean of the failure rate in component $i$ bought from supplier $j$ \\
$\sigma_{i}^{2}$ &  variance of the failure rate in component $i$ \\
$\sigma_{ij}^{2}$ & variance of the failure rate in component $i$ bought from supplier $j$ \\
$P_{n}(x_{ij})$  & probability of   failures occur in period $[0, T]$ \\
$PC(x_{ij})$ & expected penalty cost due to downtime exceeding $D_{0}$ \\
$C_f$    & average repair cost in period $[0, T]$

	\end{tabular}
  %\end{center}
	 \subsection{Assumptions}
	
   \begin{enumerate}
   \item The components in discussion are both critical components. If a component fails, the entire system will stop functioning.
  \item The system will be functioning for $T$ years. The exploitation phase is denoted by $[0, T]$. We assume that the system are sold at time $t=0$, and the acquisition cost is also incurred at time $t=0$.
	 \item During the exploitation phase of the system $[0, T]$, its total downtime should be less than or equal to $D_0$ years. If the total downtime exceeds $D_0$ years, the OEM will be charged for a penalty cost with respect to the extra downtime and the penalty rate $c_p$.
	 \item The failure rates of the components are uncertain in the design phase and distributed with the same general distribution function. Components from different rough designs have different failure rate distributions parameters and acquisition costs. Component $i$ from the rough design of supplier $j$ has the failure rate $\Lambda_{ij}$ with mean $\mu_{ij}$ and variance $\sigma_{ij}^2$ and acquisition costs $c_{ij}$.
	 \item The life time of each the component is independent and exponentially distributed. So the number of the system failures over $[0, T]$ has a Poisson distribution with the system failure rate $\Lambda$.
		\item	Once the rough design of a component has been chosen, the value of the component failure rate remains fixed before and after replaced by other new components.
  \item When a failure occurs, a repair will be performed. The failed part will be replaced by a ready-to-use part (repair by replacement)from the secondary supplier. The failure rate remains the same after the replacement. The repair cost $c_r^{ij}$ is fixed for each rough design.
 	
	\end{enumerate}
	
\subsection{Cost structures}

\subsubsection{Acquisition cost}
	
	Let $c^{ij}_{a}$ denote the cost of purchasing component $i$ from supplier $j$. Define binary decision variable $x_{ij}$ as:
	\[ x_{ij} = \left \{
	  \begin{array} {l l}
		0 & \quad \text {not purchasing component $i$ from supplier $j$} \\
		1 & \quad \text {purchasing component $i$ from supplier $j$}
		\end {array} \right.\]
		then the acquisition cost is given by:
	 \begin{eqnarray}
	A(x_{ij})=\sum ^{2}_{i=1}\sum^{2}_{j=1} {c^{ij}_{a} x_{ij}}
		 \end{eqnarray}
Assume we only select one supplier for each component, which means $\sum^{2}_{j=1}{x_{ij}=1}$.	
	
	\subsubsection{Repair cost}
	
When a failure occurs in period $[0, T]$, a repair will be performed by the OEM. The cost for each repair is $c_r^{ij}$. For a single component $i$, the expect number of repairs for $i$ during $[0,T]$, $N(x_{ij})$ is given by (Barlow and Proschan 1967):
%$E[N_{i}(T)]$
\begin {eqnarray}
N(x_{ij})=\int_0^{T}\int_{-\infty}^{\infty}{{\lambda}_{i}f_{\Lambda_{i}}({\lambda}_{i})d{\lambda}_{i}dt}=\mu_{i}T=\sum_{j=1}^{2}{\mu_{ij}x_{ij}}T
\end {eqnarray}
Where $\Lambda_{i}=\sum_{j=1}^{2}{\Lambda}_{ij}x_{ij}$. Then we have the total expected repair costs for component $i$ as:
\begin {eqnarray}
R(x_{ij})=N(x_{ij})\sum_{j=1}^{2}{c_r^{ij}x_{ij}}=\sum_{j=1}^{2}{\mu_{ij}c_r^{ij}x_{ij}}T
\end {eqnarray}

Given that the reliability function of each component is independent and exponentially distributed, the expected system repair costs in $[0,T]$ is
\begin {eqnarray}
R(x_{ij})=\sum_{i=1}^{2}\sum_{j=1}^{2}{\mu_{ij}c_r^{ij}x_{ij}}T
\end {eqnarray}

\subsubsection{Penalty cost}
When the total system downtime exceeds the predetermined value according to the PB contract, a penalty cost should be paid by the OEM to customers. The failures occur according to a Poisson process with an uncertain arrival rate $\Lambda$. The probability that $n$ failures occur during $[0, T]$ is:
\begin {eqnarray}
P_{n}(x_{ij})=\int^{\infty}_{0}{Pr\{N=n|\Lambda=\lambda\}f_{\Lambda}}(\lambda)d\lambda=\int^{\infty}_{0}\frac{e^{-\lambda T}(\lambda T)^n}{n!}f_{\Lambda}(\lambda)d\lambda
\end {eqnarray}
Given that $c_p$ is the penalty cost, $r$ is the repair time for each failure, $D_0$ is the predetermined system downtime in total, we find the expected penalty cost due to extra downtime exceeding $D_0$,
\begin {eqnarray}
PC(x_{ij})=\sum_{n=0}^{\infty}{P_{n}(x_{ij})(nr-D_{0})^{+}c_{p}}=\sum_{n=\lceil \frac{D_{0}}{r} \rceil}^{\infty}{Pr\{N=n\}(nr-D_{0})c_{p}}
\end {eqnarray}

where $(nr-D_{0})^{+}=max\{0,nr-D_0\}$.

\subsection{Problem formulation}
We formulate our problem as
\begin{eqnarray}
\text{(P)} \hspace{15mm} &\text{ min }& \hspace{10mm} \pi(\boldsymbol{x}) \nonumber\\
& \text{ s.t. }&  \hspace{5mm} \sum_{j=1}^{m}{x_{ij}}=1 \nonumber
\end{eqnarray}
where $\pi(\boldsymbol{x})$ is the expected LCC for the system and obviously, we have:
\begin{eqnarray}
\pi(\boldsymbol{x})=\sum_{i=1}^{n}{\pi(\boldsymbol{x_{i}})} \nonumber
\end{eqnarray}
and $\pi(\boldsymbol{x_{i}})$ is the expected total cost of each stage $i$, obviously,
\begin{eqnarray}
\pi(\boldsymbol{x_{i}})=A(x_{ij})+R(x_{ij})+PC(x_{ij}) \nonumber
\end{eqnarray}
\subsubsection{Decomposition into single-stage Problems}
	
   	
\end{document}
